\documentclass[11pt]{article}
\usepackage{a4,caption,listings,graphicx,enumerate,amsmath,amssymb,mathtools,lipsum,graphicx,microtype,float,xcolor, multicol, caption, fancyhdr, fourier-orns, datetime}
\usepackage[a4paper,margin=1in,footskip=0.25in]{geometry}
\usepackage[rightcaption]{sidecap}
\usepackage[sorting=none]{biblatex}


\title{\line(1,0){450} \\ \Huge\textbf{Insights through asymptotics:}\\
 \LARGE Benchmarking and analysis of climate effects on flood models}
\author{Henry Writer }

\newdateformat{monthyeardate}{%
  \monthname[\THEMONTH], \THEYEAR}

\date{\monthyeardate\today}

\usepackage{titlesec}
\renewcommand{\thesection}{\Roman{section}}

 
\titleformat{\section}[block]{\LARGE\bfseries\filcenter\titlerule\vspace{2mm}}{\thesection}{1em}{}

\titleformat{\subsection}[block]{\Large\bfseries\filcenter}{}{1em}{}
\pagestyle{fancy}
\fancyhead{}
\fancyhead[L]{Insights through asymptotics}
\fancyhead[R]{IIT-17 People and planet health}

\fancyfoot{}
\fancyfoot[C]{\thepage}
\fancyfoot[L]{Henry Writer}
\fancyfoot[R]{\monthyeardate\today}


\addbibresource{refs.bib}


\begin{document}

\maketitle

\section{Background}

There is currently unmistakable evidence that the climate of the planet is changing. We currently do not understand what the potential outcomes of these changes may be. One route we have to understand these changes takes the form of models these models then can be used to help us predict the future impact of climate change.

In the United Kingdom one of the largest potential effects of climate change we will take the form of river flooding. The government cost the 2015/16 floods at 1.6 billion pounds. This number does not take into account the human toll of this event. Additionally, it appears that the number and severity of flooding is increasing. 


This has motivated the hydrological community in increasing the understanding of river flooding. This has been done by developing a 'zoo' of potential flooding models. However, there are two problems regarding this large group of models. One since there is so many it is hard to separate them REF. 
Secondly we believe there is still information which is not being extracted from these models that can help us predict how flooding will change. We believe that we can solve both of these problems by examining models using asymptotic analysis.

\subsection{How do we model river flooding}
There are currently three main approaches to flooding model. We have statistical models this realy on using statistical method to fit models to data and use statistical models to make inferences about the future. 
We have conceptual models, these realy in creating a set of states from which water can move between this is best represented by a FIGURE. 
Finaly we have physical models these take a continuime mechanics approch and set up a set of conservation laws and use these to drive a set PDE sytems that can be used to mdoel the flooding event.



\subsection{Conceptual models}

 As briefly mentioned in the previous section conceptual models attempt to capture a complex system such as river flooding and simplify it down to a set of possible states/locations the variable of interest may reside in. Travel between these states is then governed by a simple mathematical expression.
 
 \begin{minipage}{0.5\textwidth}
    \begin{figure}[H]
        \centering
        \includegraphics[width=0.6\textwidth]{Figs/Concept.png}
        \captionof{figure}{Example of a Conceptual model}
        \label{fig:conceptual}
    \end{figure}
    \begin{figure}[H]
        \centering
        \includegraphics[width=0.6\textwidth]{Figs/EA_bench.png}
        \captionof{figure}{Ahhhhhhhhhhhhhhhhhhhhhh}
        \label{fig:eaBench}
    \end{figure}
\end{minipage}
\hspace{0.05\textwidth}
\begin{minipage}{0.4\textwidth}
    \qquad Figure \ref{fig:conceptual} gives a diagrammatic example of a conceptual model this model show how rain water might flow through a catchment area. It shows rain entering the system precipitation and how the path it might take, either evaporating immediately, or by flowing into surface or groundwater storage, before being eventually discharged.

    \qquad We can see that conceptual modelling allow us to easily build a model that in can incorporate many aspects of a system. However, with such in designing these models means that there is a `zoo' of similar models from which we could use to model a potential river catchment area. This issue of benchmarking these models to determine which one is mots suitable for a given situation are discussed in REFRENCE, 

    \qquad Figure \ref{fig:eaBench}  graphically demonstrates the challenge posed in differentiating flooding models. This is one of the major motivations of the proposed assytopic benchmarking process.
\end{minipage}







\subsection{How do we currently interpret the output of flooding models}

\begin{minipage}{0.45\textwidth}
    Another motivation to creating a methodology to derive scaling laws from conceptual models is in the predication of future flooding events. 
    The method currently used to give predictions of future flooding evolves taking a handful of different data sets that model a potential climate under a variety of possible climate situations. These data sets are then feed in to a conceptual model then these outcomes are used to draw inferences about future floods.
    This process can be visualised in the flow diagram in Figure \ref{fig:flow}.
\end{minipage}
\hspace{0.05\textwidth}
\begin{minipage}{0.45\textwidth}
    \begin{figure}[H]
    \centering
    \includegraphics[width=0.25\textwidth]{Figs/flow.png}
    \captionof{figure}{I hate the warp enviroment}
    \label{fig:flow}
\end{figure}
\end{minipage}
\vspace{5pt}

The paper by PEOPLE is a good reference for how this process is carried out in practice. Figure \ref{fig:concept_in} shows the diffreent rainfa;; data entered in to the model, and Figure\ref{fig:concept_out} show the output form there model. They then use the output form these models to make simple predictions about the effect of climate change.

We think that our assintpic benchmarking process that we can do better by deriving psychical scaling laws that will can give additional insight in to flooding.

\begin{minipage}{0.59\textwidth}
    \begin{figure}[H]
    \centering
    \includegraphics[width=\textwidth]{Figs/concept_in.jpg}
    \captionof{figure}{Mmmmmmmmmmmmmmmmmmmmmmmmmmmmmm}
    \label{fig:concept_in}
\end{figure}
\end{minipage}
\hspace{0.05\textwidth}
\begin{minipage}{0.31\textwidth}
    
\begin{figure}[H]%this figure will be at the left
    \centering
    \includegraphics[width=\textwidth]{Figs/concept_out.jpg}
    \captionof{figure}{Nooooooooooooooo}
    \label{fig:concept_out}
\end{figure}
\end{minipage}




\subsection{Reduced PDE model}
Another model of interest is a reduced PDE model. This model has been developed by a current SAMBa student Piotr as part of thesis. 
We believe that we can continue his work on this model by using the method of multiple scales to examine the behaviour of this model over different time scales.
We shall briefly outline how this model works, this brief exploration will omit many of the major details this can be found in the these paper REFRENCE.

\vspace{5pt}
\begin{minipage}{0.55\textwidth}
    \begin{figure}[H]%this figure will be at the left
    \centering
    \includegraphics[width=\textwidth]{Figs/Simple model.png}
    \captionof{figure}{hmmmmmmmmmmmmmmm}
    \label{fig:model}
    \end{figure}
\end{minipage}
\hspace{0.05\textwidth}
\begin{minipage}{0.35\textwidth}
    \qquad For this model to work we reduce the complexity of the river by assuming that it travel in a straight line, with the river laying at the bottom of a v shaped channel. Additionally, we assume there are two types of equation govering the behaviour of the flow of water toward the river. 
    The flow in the `unsaturated zone' this is where a flood has not started and the flow through the `saturated zone' this is the area where the soil has become saturated ,and a flood has formed. the equation for this system is given below,
    
\end{minipage}

 \begin{align}
    H_t=\begin{cases}
        f(x)^{-1}[(\sigma HH_x+H)_x+p_0\times\text {rainflall}] \qquad \qquad  \ \; \text{in the unsaturated zone},\\
        (\sigma H_x+\mu\sqrt{1+\sigma H_x}(H-1)^{5/3})+p_0 +\text{rainfall} \quad \text{in the saturated zone},
    \end{cases}
\end{align}

where $H(x,t)$ is the hight of the fluid at position $x$ and time $t$.





\section{Problem formulation and method}
The core of this program of work will take two forms. First we will develop this aentopitic benchmarking tool, and secondly we will explore expansions to the reduced PDE model. 
\subsection{A new approach to benchmarking}
As we discussed earlier there is work to be done in improving the insights which we can get from conceptual models. The plane will be to derive assytopic scaling laws for these conceptual models. These laws will be able to tell us how some desired quantity is scaled based upon some input. So for example from the reduced PDE model (1) we would wish to develop a scaling law of the following form, $H(r)$, where is the rainfall. 

\vspace{3pt}
\noindent\begin{minipage}{0.53\textwidth}
    \qquad However deriving the scaling laws directly is likely to be difficult so we first approach the problem numerically to demonstrate this procedure we will use the reduced PDE model. 
    
    \qquad First we pick some rainfall function in this case we we chose, \begin{align}
        r = r_0\epsilon,
    \end{align}
    where $r_0$ is some baseline rainfall. So in Figure \ref{fig:S14_fig1} each colour of line represents a different value of alpha. Next we fix some time in this case we pick time such that we are in the saturated and a flood has formed. 
    
    \qquad Next we then take these values of $H(r)$ and plot these and then attempt to fit some function to them. In this case this can be seen in Figure \ref{fig:S14_fig2} which is loglog plot form this we get that, \begin{align}
    H(r)=r_0^{\epsilon}.
    \end{align}
    
    \qquad We will confirm this by asstotpicly examining the model, unfortilay during the preliminary investigations which we carried out during ITT17 we did not have time analytical derive the scaling law for this model
    \end{minipage}
\hspace{0.05\textwidth}
\begin{minipage}{0.37\textwidth}
    \begin{figure}[H]%this figure will be at the left
    \centering
    \includegraphics[width=\textwidth]{Figs/S14_fig1.jpg}
    \captionof{figure}{hmmmmmmmmmm}
    \label{fig:S14_fig1}
    \end{figure}
    \begin{figure}[H]%this figure will be at the left
    \centering
    \includegraphics[width=\textwidth]{Figs/S14_fig2.jpg}
    \captionof{figure}{hmmmmmm}
    \label{fig:S14_fig2}
    \end{figure}
\end{minipage}

\vspace{5pt}

\qquad The two reasons why we wish to numerically, first is that it will significantly speed up the search for analytically derived laws as we will be able to use it as a ansatz. Secondly there is scope to develop the numerical work flow into a R package that will allow industry to test there models.



\subsection{Extending the model with multiple time scales}

\begin{minipage}{0.45\textwidth}
    \begin{figure}[H]
    \centering
    \includegraphics[width=\textwidth]{Figs/Rainfall.png}
    \captionof{figure}{ghsjhfj}
    \label{fig:rainfall}
    \end{figure}
\end{minipage}
\hspace{0.05\textwidth}
\indent\begin{minipage}{0.45\textwidth}
    The second strand of the proposal will focus on extending the reduced PDE model with to accommodate more complex forms of rainfall functions. Currently the model has only eb explored with constant rainfall, this is clearly not hwo the weather acts in real world see Figure \ref{fig:rainfall} which is the monthly rainfall for the UK. 
    Clearly this rainfall is not constant and in fact appears to have behaviour working on some short time scale, the seasonal difference in rainfall, and behaviour working on some larger multi year time scale.

    \qquad We should hopefully be able to apple the asstoptic mehtod of multiple time scales to improve the solutions of this problem.
\end{minipage}

\vspace{5pt}
\subsection{People}



\section{Objectives and strategy}

The work outlined above is designed to be undertaken as PhD project. A potential student undertaken this project would require a strong background in fluid mechanics, asymptotic analysis and, would require a passion for environmental mathematics. 
Additionally, it would be beneficial if a potential student had some experience with time series analysis, and numerical PDEs.

\subsection{Objectives}
Below is a list of initial targets which shall form the core of the project.

\begin{enumerate}
    \item \textbf{Numerical derive scaling laws,} from conceptual models. This would require conducting a literature review to ascertain influential models that our benchmarking method could be applied to.
    \item \textbf{Analytically derive scaling laws,} this shall be accomplished by using numerical laws derived to inform a discretion of the conceptual models. To conduct this integration we shall use asymptotic analysis.
    \item \textbf{Apply periodic rainfall to the reduced PDE model}. To this end we will need to conduct a time series analysis of rainfall data. This shall be used to find appropriate time scales to inform the analysis of the reduced PDE model.
\end{enumerate}

\subsection{Potential future objectives}

There are a number of directions this project may be extended dependent upon the interests of the student. A couple of potential avenues of extension are outlined below to show the scope of where this project.

 The reduced PDE model has a number of other extension which can be explored. One approach would be to use the astytotic method of homisation, this would allow us to explore how a more complex soil structure will effect flooding events. Another approach we could use to examine the water flow in soil would be to replace the equation which governs the flow through the soil with a conduit equation this would allow us to explore channelling behaviour in the soil.

We could also take a dynamical systems approach to analys ing the reduced PDE model. For example it very natural to assume that there is two possible states in which a catchment area could be in, a staturated state this would be equivalent to a wetland, and a dry/draught state. We then assume that as the land dry out it becomes less pouress so more watter flows direclty overland. We can then explore how much rainfall would have to change to transition between these states.

Another promesing avenuea we could explore would be attemtping to match the conceptual models with the reduced PDE model. This would be of great intrested as the different methoglgies for building flooding models are metphoricaly existe on seprated islands. By matching the PDE modle to a conceptual model we hope to be able to connect these modles to gether.

These free possible reasrech direction are not ment to be taken as a demenstartion of the scope of this projetc and it postential.



\section{Impact}
Like all research proposal we believe that we are carrying out novel and interesting mathematics. Additionally due to the subject of this research on climate change and namely river flooding which is one of a major extensional threat to the United Kingdom and the wider world. Any additional understanding of flooding mechanics and additional predictions will prove to be extremely valuable.
\subsection{Mathematical impact}
To emphasise the mathematical impact of this project here is a non-exhaustive list of the novel mathematics. 
\begin{enumerate}
    \item Applying assytopic methods to derive scaling laws for climate models is a potential new application for these methods.
    \item A deep analytical analysis of conceptual modelling methods we will help to give understanding of how these commonly used techniques will behaviour.
    \item The application of a multiple time scales analysis to the reduced PDE models is very likely to generate new and interesting problems which will need to be solved.
\end{enumerate}

\subsection{Industrial impact}
As this problem has been brought forward by the Environments Agency and ??????????? there is a clear interest in this type of work. But more concretely there is a very large motivation from practising hydrologists for a rigours analysis of the conceptual models that for the foundation of there practice. Additionally, we believe that the scaling laws that will be derived during this process may help to inform how we manage the risks of river flooding.

Finally, the second strand of this project in extending the PDE model to incorporate periodic forcing we will allow us to help examine seasonl rainfall and examine/predict chages in potential flooding over the timescale of climate change. This type of analysis is challenging for industry to carry out and is prime territory for a PhD project.






\cite{pp1}

\cite{pp2}

\cite{pp3}

\cite{BELL201289}

\printbibliography


\end{document}